\documentclass[10pt]{article}
\usepackage{../_macrosLatex/macros}
\usepackage[a4paper,margin=2.5cm]{geometry}
\usepackage{fancyhdr} % Gestion header/footer

%----------------------------------------------------
% Paramétrage de la fiche
%----------------------------------------------------
\matiere{BTS SNIR TS1}
\sequence{Programmation en langage C++}
\seqlogo{\faCode}
\titrefiche{TP3 - Structures répétitives}
\version{v1.0}
\dateversion{05.10.22}
\type{TP}

%----------------------------------------------------
% Définition des pieds et têtes de page
%----------------------------------------------------
% Pour toutes les pages
\pagestyle{fancy}
\fancyhead[L]{\seqlogo\ \sequenceVal}
\fancyhead[R]{\titreficheVal}
\fancyfoot[L]{\matiereVal - Lycée Louis Rascol, Albi}
\fancyfoot[R]{\ccbyncsaeu}
\fancyfoot[C]{\thepage\ / \pageref{LastPage}}
% Pour la première page
\fancypagestyle{firstpage}{%
  \lhead{}
  \rhead{}
  \renewcommand{\headrulewidth}{0pt}
}

%----------------------------------------------------
% Début du document
%----------------------------------------------------
\begin{document}
\cartouche
\thispagestyle{firstpage}

\section{Introduction}
Dans ce nouveau TP nous allons utiliser les structures répétitives vues en cours telles que : \mintinline{cpp}`while()`,
\mintinline{cpp}`do ... while` et \mintinline{cpp}`for()`.

\begin{enumerate}
    \item Chaque exercice devra être \textbf{codé dans un projet CLion différent}.
    \item \textbf{Le code résidera dans le fichier main.cpp}
    \item Pour commencer copiez-coller le squelette de base d'un code C++ \textbf{dans le fichier main.cpp}.
    \item Remplissez correctement le cartouche pour chaque exercice.
\end{enumerate}

\begin{cppcode}
    /************************************************
    Nom du fichier : 
    Description : 
    Auteur :
    Date :
    BTS SNIR - TS1
    ************************************************/

    #include <iostream>
    using namespace std;

    int main() {
        // Mettez ici votre code C++
        return 0;
    }
\end{cppcode}


%----------------------------------------------------
% EXERCICES
%----------------------------------------------------
\section{Exercices}


%------------------
% EXO
%------------------
\subsection{Calculatrice bis}
Reprenez l'exercice de la calculatrice du TP2 et donnez la possibilité à l'utilisateur de rentrer à nouveau 2 nombre après lui affiché le résultat du calcul, l'utilisateur pourra quitter à tapant \mintinline{cpp}`0` lors de la demande du choix de l'opérateur. 

%------------------
% EXO
%------------------
\subsection{Somme des entiers naturels}

Dans cet exercice, il s'agit de faire la somme des entiers naturels, soit la somme des nombres entiers positifs : 1,2,3,4 ...\\
Votre programme demandera à l'utilisateur de saisir un entier \mintinline{cpp}`n` qui représentera le nombre d'entiers naturels à additionner et stockera le résultat dans la variable \mintinline{cpp}`sum` avant de l'afficher sur la console. 

\smallskip
Par exemple avec \mintinline{cpp}`n=5`:\\
$$sum=1+2+3+4+5$$

\begin{enumerate}
    \item En C++ déclarez et initialisez à 0 les variables qui contiendront la somme des entiers \mintinline{cpp}`sum` et le nombre d'entiers voulu \mintinline{cpp}`n` .
    \item Afficher sur la console la demande d'un entier à l'utilisateur et gérez la capture de cet entier au clavier pour le stocker dans \mintinline{cpp}`n`.
    \item Créez une structure \mintinline{cpp}`for()`, qui permettra de tourner dans la boucle le nombre de fois désigné par \mintinline{cpp}`n` .
    \item Dans la boucle \mintinline{cpp}`for` trouvez l'instruction permettant d'additionner la somme précédente avec l'entier actuel et de mettre le résultat dans \mintinline{cpp}`sum`.
    \item Pour terminer affichez \mintinline{cpp}`sum` à l'écran.
\end{enumerate}


%------------------
% EXO
%------------------
\subsection{Multiples et non multiples}

Combien y a-t-il d'entiers entre 1 et 20000 qui sont multiples de 7, mais pas multiples de 9 ?\\
Le nombre à trouver est 2540.

\begin{enumerate}
    \item En C++ déclarez et initialisez à 0 la variable de comptage \mintinline{cpp}`cpt`  qui contiendra l'\textbf{entier} recherché.
    \item Créez une structure \mintinline{cpp}`for()`, qui permettra de tourner dans la boucle le nombre de fois désiré.
    \item Dans la boucle \mintinline{cpp}`for` ajouter une structure \mintinline{cpp}`if` afin de faire le test demander.
    \item Si le test est positif incrémentez la variable de codage \mintinline{cpp}`cpt`.
\end{enumerate}


%------------------
% EXO
%------------------
\subsection{Calcul de factorielle}

La factorielle d'un nombre est le produit de tous les entiers à partir de 1 jusqu'à ce nombre. La factorielle peut seulement être
définie pour des \textbf{entiers positifs}.\\ La factorielle d'un nombre négatif n'existe pas et la \textbf{factorielle de 0 est 1}.

\smallskip
Par exemple :\\
La factorielle de 5 est définie par l'expression mathématique $5!$ et vaut :
$$5!=1 \times 2 \times 3 \times 4 \times 5 = 120$$

\smallskip
Créez un programme qui demande à l'utilisateur de saisir un entier \mintinline{cpp}`n`, qui vérifiera si $n\geq 0$ et calculera sa factorielle, dans le cas contraire un message d'erreur sera affiché et il sera demandé à l'utilisateur de saisir un nouveau nombre. Le résultat du calcul sera affiché sur la console.

%------------------
% EXO
%------------------
\subsection{Suite de Fibonacci}

On considère la suite de nombres entiers dont les deux premiers termes sont :\\
$$F_0=1 \quad F_1=1$$
Les termes suivants de cette suite sont construits de la manière suivante :\\
$$F_n=F_{n-1}+F_{n-2}$$
Écrire un code qui à partir d'un entier $n\geq 0$ affiche le n-ième terme de la suite.\\

\medskip
Par exemple selon la valeur de n :
$$3\rightarrow 2$$
$$11\rightarrow 89$$
$$16\rightarrow 987$$

\end{document}